\documentclass[addpoints]{exam}

\usepackage[italian]{babel}
\usepackage[utf8]{inputenc}

\pagestyle{headandfoot}
\firstpageheadrule
\firstpageheader{Informatica}{2° Assignment}{10 Marzo 2022}
\firstpagefootrule
\firstpagefooter{}{Pagina \thepage\ di \numpages}{}
\runningheadrule
\runningheader{Informatica}{2° Assignment}{10 Marzo 2022}
\runningfootrule
\runningfooter{}{Pagina \thepage\ di \numpages}{}

\pointpoints{punto}{punti}

\begin{document}
    \section{Assignment}
    Sei stato preso come dipendente all'ospedale e sei stato incaricato di scrivere un programma che permetta l'inserimento dei pazienti nel sistema.

    Il software deve poter effettuare le seguenti operazioni:
    \begin{enumerate}
        \item All'avvio, viene presentato un menù che permette di svolgere le opeazioni descritte successivamente, siete liberi di scegliere come preferite i vari selettori del menù, se usare i numeri o i caratteri. Tuttavia il menù deve prevedere che alla selezione di 'q' deve uscire dall'applicazione;
        \item Deve poter permettere l'inserimento di un paziente che contiene i seguenti dati: nome, cognome, codice fiscale, data di nascita, via di residenza e cap di residenza. Naturalmente l'inserimento di un paziente non vede la selezione di "dove" salvarlo all'interno di un array, dovete farlo automaticamente (abbiamo già visto come);
        \item Ogni paziente ha assegnato un identificativo \textbf{UNICO}. Potete usare un intero che incrementa a ogni inserimento dell'utente (quindi qualcosa che l'utente non inserice, ma ci pensa il programma);
        \item Deve poter cancellare un paziente dal sistema, create un sistema che permetta la ricerca e eliminazione per codice fiscale dell'utente;
        \item Riciclando il sistema ricerca usato al punto precedente, utilizzatelo per la modifica.
    \end{enumerate}

    Vi consiglio vivamente di attendere la prossima settimana per svolgere le operazioni sui dati. Potete in questa settimana intanto svolgere tutta la parte di gestione del menù.

    \section{Consegna}
    \label{consegna}
    La consegna deve seguire in maniera rigida i seguenti punti:
    \begin{itemize}
        \item Chiamare il progetto di Code::Blocks \texttt{CognomeNome\_Assignment\_1};
        \item Dovete effettuare lo zip del vostro progetto \textbf{SENZA I SEGUENTI FILE/CARTELLE} (altrimenti non sarò in grado di valutarlo): 
        \begin{itemize}
            \item la cartella \texttt{bin};
            \item la cartella \texttt{obj};
            \item i file con estensione \texttt{.depend};
            \item i file con estensione \texttt{.layout}.
        \end{itemize}
        \item Rinominare lo zip con lo stesso nome del progetto \texttt{CognomeNome\_Assignment\_1};
        \item Caricare lo zip su Classroom \textbf{PRIMA} della scadenza.
    \end{itemize}

    \clearpage
    
    \section{Valutazione}
    La valutazione del progetto sarà effettuata assegnando punti secondo le seguenti categorie:
    \begin{itemize}
        \item Consegna effettuata secondo i termini previsti alla sezione \ref{consegna};
        \item Il progetto compila correttamente e non ha warnings;
        \item Il software esegue correttamente se l'utente inserisce come input i valori richiesti;
        \item Il software esegue correttamente se l'utente, impazzendo, inserisce valori randomici ove c'è input (sempre nel limite del: se chiedo un intero, inserirò un intero);
        \item Il software presenta funzioni e ulteriori file, scorporando dal main.cpp le operazioni chiave del programma. Naturalmente \textbf{senza} esagerare;
        \item Il software è scritto in inglese (anche correggiuto), sono esenti gli output;
        \item Il codice sorgente è ordinato e scritto bene;
    \end{itemize}
\end{document}